  % table?
  Important heat limits in materials of the repository restrict loading designs 
  and capacity.
  %        File: heat_tab.tex
%     Created: Thu Aug 04 11:00 AM 2011 C
% Last Change: Thu Aug 04 11:00 AM 2011 C
%
\begin{table}[h!]
  \centering
  \footnotesize{
  \begin{tabular}{|l|r|r|r|r|}
    \multicolumn{5}{c}{\textbf{Thermal Behavior of Various Concepts}}\\
    \hline
    Feature & Clay & Granite & Salt & Deep Borehole \\ 
            & (Bentonite & (Concrete & (Salt & (Bentonite\\ 
            & Buffer) & Buffer) & Backfill) & Buffer) \\ 
    \hline
    Buffer Limit $[^{\circ}C]$ & 100  & 100  & 180 & 100  \\ 
    Reference
    & \cite{hardin_generic_2011}   
    & \cite{von_lensa_red-impact_2008}   
    & \cite{von_lensa_red-impact_2008,brewitz_long-term_2002,carter_thermal_2011}   
    & \cite{von_lensa_red-impact_2008}  \\ 
    &      &      &     &      \\
    Host Limit $[^{\circ}C]$   & 100  & 200  & 180 & none \\ 
    Reference                     
    & \cite{<++>}   
    & \cite{<++>}   
    & \cite{<++>}   
    & \cite{<++>}   \\
    &      &      &     &      \\
    $\alpha_{th} [\frac{10^{-6}m^2}{s}]$ & $0.12-0.19$ & $0.9-1.8$ & $1.3-2.1$ & $0.9-1.8$ \\ 
    Reference                     
    & \cite{tikhonravova_effect_2007} 
    & \cite{durham_thermal_1987,hardin_generic_2011,kim_thermal_2007}     
    & \cite{hardin_generic_2011,nieland_storage_2001}   
    & \cite{durham_thermal_1987,hardin_generic_2011,kim_thermal_2007}   \\ 
    &      &      &     &      \\
    $K_{th} [\frac{W}{m{\cdot}K}]$ & $1-2$ & $2-4$ & $\sim4$  & $2-4$ \\ 
    Reference                     
    & \cite{hardin_generic_2011,tikhonravova_effect_2007}    
    & \cite{hardin_generic_2011,kim_thermal_2007,surma_porosity_2003,ab_long-term_2006}    
    & \cite{hardin_generic_2011,nieland_storage_2001}
    & \cite{hardin_generic_2011,kim_thermal_2007,surma_porosity_2003}\\ 
    &      &      &     &      \\
    Coalesence & yes & no & yes & no \\ 
    \hline
  \end{tabular}
  \caption[Models for Heat Transport for Various Geologies]{Maximum heat load constraints, thermal 
  diffusivities, and thermal conductivities vary among repository concepts and host formations. }
  }
  \label{tab:heat_tab}
\end{table}


  Similar heat transport models can be used for all geologies, but are 
  differentiated by material parameters $(c_p, K, \rho)$ and varying 
  thermal limits.

To inform dynamic behavior within the simulator, the repository requires 
a transient model capable of quickly arriving at a heat based 
capacity for an arbitrary waste stream. 
\begin{figure}[htp]
  \begin{center}
    \includegraphics[width=0.7\columnwidth]{./images/sinkfacility.eps}
  \end{center}
  \caption{\footnotesize{The Cyder repository model has the same interface with the simulation 
  as does a sink facility. It receives materials according to some capacity. The 
  heat-limited capacity of the repository will be reassessed for new waste 
  streams offered to the repository.}}
  \label{fig:cydersink}
\end{figure}

\subsubsection{Specific Temperature Change Method}
Introduced by Radel, Wilson et. al., the Specific Temperature Change method uses 
a linear approximation to arrive at the thermal loading density limit.  
When the thermal time constant of the rock is much shorter than the waste form 
decay package, the change in package wall temperature can be described by 

\begin{align}
q(t_0)\rho_{limit}C'&=\Delta T_1
\intertext{where}
\rho_{limit} &= \frac{C_1}{Q_1}\nonumber\\
C' &= \mbox{ Thermal constant }[-]\nonumber\\
\Delta T &= T_{lim}-T_{amb}[^{\circ}C]\nonumber\\
T_{lim} &= \mbox{ Temperature limit }[^{\circ}C]\nonumber\\
T_{amb} &= \mbox{ Ambient rock temperature }[^{\circ}C]\nonumber
\end{align}


\begin{figure}[htp!]
\begin{center}
\includegraphics[width=0.8\columnwidth]{images/fakeArbitraryWF.eps}
\end{center}
\caption{\footnotesize{As a demonstration of the calculation procedure, the specific 
temperature change curves, $\Delta t$, are calculated for heat contributing 
isotopes at a 
specified repository spacing, $s$, heat limit radius, $r_{lim}$, and thermal paramters 
$\alpha_{th}$ and $K_{th}$. The total temperature change is the sum of the 
mass scaled curves $\Delta T$.}}
\label{fig:fakeArbitraryWF}
\end{figure}

Repeated runs of a detailed analytic model have given a dataset that 
facilitates estimation of thermal loading capacity for a range of thermal 
diffusitivities, conductivities, and repository layouts.
\begin{table}[ht!]
\centering
\footnotesize{
\begin{tabular}{|l|l|l|r|}
\multicolumn{4}{c}{\textbf{Thermal Cases}}\\
\hline
\textbf{Parameter} & \textbf{Symbol} & \textbf{Units} & \textbf{Value Range} \\
\hline
Thermal Diffusivity & $\alpha_{th}$ & $[m^2\cdot s^{-1}]$ & $1.0\times10^{-7}-3.0\times10^{-6}$\\
\hline
Thermal Conductivity & $K_{th}$     & $[W\cdot m^{-1} \cdot K^{-1}]$ & $0.1 - 4.5$ \\
\hline
Grid Spacing & $S$ & $[m]$ & 2, 5, 10, 15, 20, 25, 50 \\
\hline
Calculation Radius & $r$ & $[m]$ & 0.1, 0.25, 0.5, 1, 2, 5 \\
\hline
Isotope & $i$ & $[-]$ & $^{241,243}Am,$  \\
        & & & $^{242,243,244,245,246}Cm,$  \\
        & & & $^{238,240,241,242}Pu$  \\
\hline
\end{tabular}
\caption{A lookup table of Specific Temperature Change values as a function of 
each of these paramters was generated by repeated parameterized runs of the LLNL 
MathCAD model\cite{greenberg_application_2012, sutton_investigations_2011}.}
\label{tab:therm_cases}
}
\end{table}



% LLNL

\begin{frame}[ctb!]
\frametitle{LLNL Model : Background}
The analytical  model
\begin{itemize} 
  \item was created at LLNL (H. Greenberg, J. Blink, et. al) \cite{hardin_generic_2011, sutton_investigations_2011, 
greenberg_application_2012}
  \item employs an analytic model from Carslaw and Jaeger \cite{carslaw_conduction_1959} 
  \item is implemented in MathCAD \cite{ptc_mathcad_2010}
  \item seeks to inform heat limited waste capacity calculations for 
    \begin{itemize}
      \item arbitrary geology 
      \item arbitrary waste package loading densities
      \item arbitrary homogeneous decay heat source
    \end{itemize}
\end{itemize}
\end{frame}

\begin{frame}
  \frametitle{LLNL Model : Geometry}
  \begin{figure}[h!]
    \begin{center}
      \includegraphics[width=0.7\columnwidth]{images/llnlConcept.eps}
    \end{center}
    \caption{Vertical, horizontal, alcove, and borehole emplacement layouts can 
    be represented by a line of point sources and adjacent line sources 
    \cite{sutton_investigations_2011}.}
    \label{fig:llnl}
  \end{figure}
\end{frame}


