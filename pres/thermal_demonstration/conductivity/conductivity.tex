

\begin{frame}[ctb!]
\frametitle{LLNL Model Thermal Conductivity Sensitivity}
\begin{figure}[htbp!]
\begin{center}
\includegraphics[height=0.7\textheight]{./thermal_demonstration/conductivity/Cm242kth_alpha_low.eps}
\end{center}
\caption[$K_{th}$ Sensitivity for Low $\alpha_{th}$ in LLNL Model]{
By varying the thermal conductivity of the repository model from 0.1 to 4.5 
$[W\cdot m^{-1} \cdot K^{-1}]$, this sensitivity analysis succeeds in capturing 
the domain of thermal conductivities witnessed in high thermal conductivity 
salt deposits as well as low thermal conductivity clays.}
\label{fig:Cm242Kth_alpha_low}
\end{figure}

\end{frame}


\begin{frame}[ctb!]
\frametitle{Cyder Thermal Conductivity and Limiting Radius Sensitivity}


\begin{figure}[htbp!]
\begin{center}
\includegraphics[height=0.7\textheight]{./thermal_demonstration/conductivity/kr.eps}
\end{center}
\caption[$K_{th}$ vs. $r_{lim}$ Sensitivity in Cyder]
{
Increased thermal conductivity of a medium decreases thermal energy deposition 
in the near field. Additionally, analysis with the \Cyder STC database 
demonstrates the way in which the importance of spacing and the importance of 
the limiting radius decrease with increasing $K_{th}$.
The above example thermal profile results from 10kg of 
$^{242}Cm$}
\label{fig:kr}
\end{figure}
\end{frame}

\begin{frame}[ctb!]
\frametitle{Cyder Thermal Conductivity and Limiting Radius Sensitivity}

\begin{figure}[htbp!]
\begin{center}
\includegraphics[height=0.7\textheight]{./thermal_demonstration/conductivity/ks.eps}
\end{center}
\caption[$K_{th}$ vs. Waste Package Spacing Sensitivity in Cyder]{
The combined effect of waste package spacing and $K_{th}$ is strong. The above example thermal profile results from 10kg of 
$^{242}Cm$}
\label{fig:ks}
\end{figure}
\end{frame}


