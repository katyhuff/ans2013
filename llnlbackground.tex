\subsubsection{Supporting Thermal Response Dataset}

Repeated runs of a detailed analytic model over the range of values in Table 
\ref{tab:thermal_cases} have given a dataset that 
facilitates estimation of thermal loading capacity for a range of thermal 
diffusivity values, conductivity values, and repository layouts. Simple linear 
interpolation across the discrete parameter space provides a simple thermal 
reference dataset for use in Cyder.

\begin{table}[ht!]
\centering
\footnotesize{
\begin{tabular}{|l|l|l|r|}
\multicolumn{4}{c}{\textbf{Thermal Cases}}\\
\hline
\textbf{Parameter} & \textbf{Symbol} & \textbf{Units} & \textbf{Value Range} \\
\hline
Diffusivity & $\alpha_{th}$ & $[m^2\cdot s^{-1}]$ & $1.0\times10^{-7}-3.0\times10^{-6}$\\
\hline
Conductivity & $K_{th}$     & $[W\cdot m^{-1} \cdot K^{-1}]$ & $0.1 - 4.5$ \\
\hline
Spacing & $S$ & $[m]$ & 2, 5, 10, 15, 20, 25, 50 \\
\hline
Radius & $r_{lim}$ & $[m]$ & 0.1, 0.25, 0.5, 1, 2, 5 \\
\hline
Isotope & $i$ & $[-]$ & $^{241,243}Am,$  \\
        & & & $^{242,243,244,245,246}Cm,$  \\
        & & & $^{238,240,241,242}Pu$  \\
        & & & $^{134,135,137}Cs$  \\
        & & & $^{90}Sr$  \\
\hline
\end{tabular}
\caption{A thermal reference dataset of \gls{STC} values as a function of each of these parameters was generated by repeated parameterized runs of the LLNL 
MathCAD model\cite{greenberg_application_2012, greenberg_investigations_2012}.}
\label{tab:thermal_cases}
}
\end{table}



The analytic model used to populate the reference dataset was created at 
\gls{LLNL} for the \gls{UFD} campaign. In this tool, heat limited thermal 
response is calculated analytically for each geology, for many waste package 
loading densities, and for many fuel cycle options \cite{hardin_generic_2011, 
greenberg_investigations_2012, greenberg_application_2012}. It employs an 
analytic model from Carslaw and Jaeger and is implemented in MathCAD 
\cite{carslaw_conduction_1959, ptc_mathcad_2010}.  The integral solver in the 
MathCAD toolset is the primary calculation engine for the analytic MathCAD 
thermal model, which relies on superposition of point, finite-line, and line 
source integral solutions.  

%The transient state of the temperature at the calculation radius is found with a convolution of the transient far field solution with the steady state near field solution.  The process is then iterated with a one year resolution in order to arrive at a temperature evolution over the lifetime of the repository. 
%
%In a two dimensional grid of waste packages, the central package is represented by the finite line solution
%\begin{align}
%  T_{line}(t,x,y,z) &= \frac{1}{8\pi K_{th}} 
%  \bigintsss_0^t\!\frac{q_L(t')}{t-t'}e^{ \frac{-\left(x^2 + z^2\right)}{4\alpha 
%  (t-t')} }\nonumber\\ &\cdot\left[ \erf{\left[ \frac{1}{2} \frac{\left( y + 
%  \frac{L}{2} \right)}{\sqrt{\alpha(t-t')}}  \right]} - \erf{\left[ \frac{1}{2} 
%  \frac{\left( y - \frac{L}{2} \right)}{\sqrt{\alpha(t-t')}}  \right]} 
%  \right]\,\mathrm{dt'},
%  \label{line}
%  \intertext{adjacent packages within the central tunnel are represented by the 
%  point source solution }
%  T_{point}(t,r) &= 
%  \frac{1}{8K_{th}\sqrt{\alpha}\pi^{\frac{3}{2}}}\bigintsss_0^{-t}\!\frac{q(t')}{(t-t')^{\frac{3}{2}}}e^{\frac{-r^2}{4\alpha(t-t')}}\,\mathrm{dt'},
%  \label{point}
%  \intertext{and adjacent disposal tunnels are represented by infinite line 
%  source solutions}
%  T_{\infty line}(t,x,z) &= \frac{1}{4\pi K_{th}} 
%  \bigintsss_0^t\!\frac{q_L(t')}{t-t'}e^{ \frac{-\left(x^2 + z^2\right)}{4\alpha 
%  (t-t')} }
%  \intertext{in infinite homogeneous media, where}
%  \label{infline}
%  \alpha &= ~~\mbox{thermal diffusivity } [m^2\cdot s^{-1}]\nonumber\\
%  q(t) &= ~~\mbox{point heat source} [W]\nonumber\\
%  \intertext{and}
%  q_L(t) &= ~~\mbox{linear heat source} [W\cdot m^{-1}]\nonumber
%\end{align}
%Superimposed point and line source solutions allow for a notion of the 
%repository layout to be modeled in the host rock.
