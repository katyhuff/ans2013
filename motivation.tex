% Provide a brief description of the importance of the work (what problem it 
%   addresses/solves):

The United States is simultaneously considering a number of domestic nuclear 
fuel cycle and geologic disposal options \cite{doe_strategy_2013}.  These decisions are technologically 
coupled by repository capacity. That is, the thermal capacity of a geologic 
repository is a strong function of site geology thermal parameters (see Table 
\ref{tab:heat_tab}) as well as spent fuel composition, which varies among 
alternative fuel cycles. 

To inform research and development in this coupled system, a generic geologic disposal 
performance model capable of dynamic integration with a systems analysis 
framework is necessary to illuminate capacity constraints and dynamic feedback 
effects of candidate repository geologies in the context of fuel cycle options.
In answer to this need, the algorithm in this work has been implemented in the 
Cyder software library which integrates with the Cyclus computational 
fuel cycle systems analysis platform \cite{huff_cyder_2013,wilson_cyclus:_2012}. 

%        File: heat_tab.tex
%     Created: Thu Aug 04 11:00 AM 2011 C
% Last Change: Thu Aug 04 11:00 AM 2011 C
%
\begin{table}[h!]
  \centering
  \footnotesize{
  \begin{tabular}{|l|r|r|r|r|}
    \multicolumn{5}{c}{\textbf{Thermal Behavior of Various Concepts}}\\
    \hline
            & Clay & Granite & Salt & Deep \\ 
            & & & & Borehole \\ 
            & (Bentonite & (Concrete & (Salt & (Bentonite\\ 
            & Buffer) & Buffer) & Backfill) & Buffer) \\ 
    \hline
    Buffer Limit $[^{\circ}C]$ & \textbf{100}  & \textbf{100}  & \textbf{180} & \textbf{100}  \\ 
    Reference
    & \cite{hardin_generic_2011}   
    & \cite{von_lensa_red-impact_2008}   
    & \cite{von_lensa_red-impact_2008,brewitz_long-term_2002}   
    & \cite{von_lensa_red-impact_2008}  \\ 
    &      &      &     &      \\
    Host Limit $[^{\circ}C]$   & \textbf{100}  & \textbf{200}  & \textbf{180} & \textbf{none} \\ 
    Reference                     
    & \cite{andra_argile:_2005}   
    & \cite{von_lensa_red-impact_2008}   
    & \cite{hardin_generic_2011}   
    & \cite{hardin_generic_2011, brady_deep_2009}   \\
    &      &      &     &      \\
    $\alpha_{th} [\frac{10^{-6}m^2}{s}]$ & \textbf{0.12-0.19} & \textbf{0.9-1.8} & \textbf{1.3-2.1} &\textbf{ 0.9-1.8} \\ 
    Reference                     
    & \cite{tikhonravova_effect_2007} 
    & \cite{durham_thermal_1987,hardin_generic_2011,kim_thermal_2007}     
    & \cite{hardin_generic_2011,nieland_storage_2001}   
    & \cite{durham_thermal_1987,hardin_generic_2011,kim_thermal_2007}   \\ 
    &      &      &     &      \\
    $K_{th} [\frac{W}{m{\cdot}K}]$ & \textbf{1-2} & \textbf{2-4} & $\mathbf{\sim4}$  & \textbf{2-4} \\ 
    Reference                     
    & \cite{hardin_generic_2011,tikhonravova_effect_2007}    
    & \cite{hardin_generic_2011,kim_thermal_2007,surma_porosity_2003,ab_long-term_2006}    
    & \cite{hardin_generic_2011,nieland_storage_2001}
    & \cite{hardin_generic_2011,kim_thermal_2007,surma_porosity_2003}\\ 
    &      &      &     &      \\
    Coalescence & yes & no & yes & no \\ 
    \hline
  \end{tabular}
  \caption[Models for Heat Transport for Various Geologies]{Maximum heat load constraints, thermal 
  diffusivity, and thermal conductivities vary among repository concepts and host formations. }
  \label{tab:heat_tab}
  }
\end{table}


A generic repository model appropriate for systems analysis must emphasize 
modularity and speed while providing modeling options at various levels of 
detail. Therefore, parameterized simulations and abstraction efforts conducted to develop 
the method described in this work sought to capture the dominant physics of 
thermal repository capacity assessment so that the Cyder disposal environment 
library could meet the simulation speed requirements of the Cyclus fuel cycle 
simulator.

