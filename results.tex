% Provide your results:
%       clearly

The primary outcome of this work is a mulitdimensional database of repository temperature 
change per mass of high heat contributing isotopes supporting the implementation 
of the Specific Temperature Change Method in Cyder. 

A validation effort concerning this tool was performed to assess the validity of 
the Specific Temperature Change method for the purpose of repository thermal 
response estimation. Comparison of the results of this method with the \gls{LLNL} model gave 
appropriately accurate results and demonstrated the way in which inaccuracies 
are bounded. Figure \ref{fig:CmValidation} shows the results of an example validation exercise comparing the combined scaling and  
superposition calculations demonstrated in Figures \ref{fig:CmScaling} and 
\ref{CmSuperposition} respectively.

\begin{figure}[htp!]
\begin{center}
\includegraphics[width=\columnwidth]{images/CmValidation.eps}
\end{center}
\caption{The validation test for the combined $Cm$ inventory per MTHM in 51GWd burnum UOX PWR fuel compares favorably with. }
\label{fig:CmValidation}
\end{figure}

% Unit Test Results
In addition to this validation effort, continual verification of code behavior
is enabled by a suite of unit tests packaged with the tool. These tests may
continually be performed to evaluate the implementated behavior of units of
functionality within the interpolation and specific temperature change
algorithms even as the code is improved in the future.  

% Base Case Demonstration ???
The base case demonstration of integration with the Cyclus next generation 
fuel cycle simulator.
