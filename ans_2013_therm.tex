%        File: ans_2013_therm.tex
%     Created: Mon Aug 13 10:00 AM 2012 C
%

%% To use the glossaries acronym package, you'll need to define any acronyms you intend to 
%% use. You can define acronyms with \newacronym{label}[acronym]{written out form}
%% To refer to them in the text use \gls{label}

\documentclass{anstrans}
%%%%%%%%%%%%%%%%%%%%%%%%%%%%%%%%%%%
\usepackage[acronym,toc]{glossaries}
\makeglossaries

% concise yet adequately descriptive title
\title{Rapid Determination of Thermal Repository Capacity For Fuel Cycle Analysis}
\author{Kathryn D.~Huff$^1$, Alexander T. Bara$^2$}

%% uncomment these next five only if using anstrans
\institute{$^1$Univ. of Wisconsin, 1500 Engineering Dr., Madison, WI, 53706\\ 
\& Argonne National Laboratory, 9700 S. Cass Ave., Lemont, IL, khuff@anl.gov\\
$^2$Univ. of Illinois, Urbana Champaign, IL, 61801, bara1@illinois.edu}
\usepackage{graphicx}
\usepackage{booktabs} % nice rules for tables
\usepackage{microtype} % if using PDF
\newcommand{\units}[1] {\:\text{#1}}%
\newcommand{\SN}{S$_N$}%{S$_\text{N}$}%{$S_N$}%

\date{}
%%%%%%%%%%%%%%%%%%%%%%%%%%%%%%%%%%%
\begin{document}
%%%%%%%%%%%%%%%%%%%%%%%%%%%%%%%%%%%%%%%%%%%%%%%%%%%%%%%%%%%%%%%%%%%%%%%%%%%%%%%%
\section{Introduction}
% Provide a summary of the work conducted:
%      Describe the technical problem clearly
%      support it with a method

A method of rapid thermal repository capacity calcuation for use in Cyder, a 
software library for coupled thermal and hydrologic repository performance 
analysis is described. Integration of Cyder with the Cyclus fuel cycle simulator 
is also described. A proof of principle demonstration is also presented, in 
which the rapid calculation method described here is benchmarked against results 
of more detailed models.

This method employs a specific temperature change method \cite{radel} and has 
resulted from combining detailed spent nuclear fuel composition data with 
detailed thermal repository performance analysis 
tools from \gls{LLNL} and \gls{ANL}\cite{radel,llnl,sinda}.

Abstraction of detailed computational thermal repository performance model 
results has resulted in a dataset and associated specific temperature change 
estimation algorithm. This algorithm is capable of rapid estimation of temperature increase near 
emplacement tunnels as a function of waste composition, near field thermal 
conductivity, and near field thermal diffusitivity.  This method has been implemented in Cyder, 
a software library of interchangeable 
radionuclide transport models appropriate for representing natural and 
engineered barrier components of generic geology repository concepts.
Cyder is an open source library intended for integration with the 
Cyclus Fuel Cycle Simulator, but with an interface suitable for linking to other 
tools or for use as a stand-alone repository behavior calculation engine. 

The Cyder software library integrates with 
the Cyclus computational fuel cycle systems analysis platform in order to 
calculate repository performance metrics with respect to candidate fuel cycle 
options \cite{huff_cyder_2012,huff_cyclus:_2010}. By abstraction of more 
detailed thermal models, Cyder aims to capture the dominant 
physics of thermal phenomena affecting repository performance in 
various geologic media and as a function of spent fuel composition.

\section{Motivation}
% Provide a brief description of the importance of the work (what problem it 
%   addresses/solves):
Thermal evolution of a geologic repository is a strong function 
of spent fuel composition, which varies among alternative fuel cycles. For this 
reason, a generic disposal model capable of dynamic integration with a systems analysis 
framework is necessary to illuminate capacity constraints and dynamic 
feedback effects of candidate repository geologies in the context of fuel cycle options.

A generic repository model appropriate for systems analysis must emphasize 
modularity and speed while providing modeling options at various levels of 
detail. Parameterized simulations and abstraction efforts conducted to develop 
the method described in this work sought to capture the dominant physics of 
thermal repository capacity assessment so that the Cyder disposal environment 
library could meet the simulation speed required by the Cyclus fuel cycle 
simulator.


\section{Results and Analysis}
% Provide your results:
%       clearly

The result of this work is a mulitdimensional database of repository temperature 
change per mass of high heat contributing isotopes 

consist of four models that are the product of an 
abstraction effort with more detailed tools. Results of unit tests and 
benchmarking efforts will be described as will a proof of principle base case 
demonstration of the use of these models. The base case concept has used these 
tools to model a generic, isotropic, permeable porous geological medium with 
reducing geochemistry as well as waste form, waste package, and buffer models in 
the near field.

The analytic models modified by abstraction and implemented in Cyder include a 
degradation rate model, a mixed cell model, a response function model, and a 
one dimensional solution (Brenner, 1962) to the advection-dispersion equation.

The degradation rate model simulates the fractional degradation of the material 
containment properties is the simplest of implemented models and is most 
appropriate for simplistic waste package failure modeling. 

Slightly more complex, and suited to representating waste form and buffer 
components, the mixed cell model incorporates solubility limited, congruent 
release under the influence of elemental partitioning coefficients and diffusion 
coefficients as well as advective velocity. Abstraction results concerning the 
transition between primarily diffusive and primarily advective transport regimes 
was used for benchmarking and iterative accuracy improvements in the development 
of this model.

The lumped parameter, response function model implemented interchangeable piston flow, 
exponential, and dispersion response functions and was developed by direct 
calibration against the results of the abstraction effort.  

Finally, abstraction results informed modifications implementation of an 
analytic solution to the one dimensional advection-dispersion equation with 
finite domain and Cauchy and Neumann boundary conditions at the inner and outer 
boundaries, respectively. 

\section{Conclusions}
% Was: Importance to Others - Replace with a conclusion.
The Cyder source code in which these models are implemented as well as 
associated documentation are freely available for use by model developers in the 
field of nuclear waste management. The application programming interface to this 
software library is intentionally general to facilitate the incorporation of the 
models presented here within software tools in need of a multicomponent repository 
model.

Furthermore, this work contributes to an expanding ecosystem of computational 
models available for use with the Cyclus fuel cycle simulator. This hydrologic 
nuclide transport library, by virtue of its capability to modularly integrate 
with the Cyclus fuel cycle simulator has laid the foundation for integrated 
disposal option analysis in the context of fuel cycle options. 

%%%%%%%%%%%%%%%%%%%%%%%%%%%%%%%%%%%%%%%%%%%%%%%%%%%%%%%%%%%%%%%%%%%%%%%%%%%%%%%%
%\nocite{*}
\bibliographystyle{ans}
\bibliography{bibliography}
\end{document}



